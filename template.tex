% Options for packages loaded elsewhere
% Options for packages loaded elsewhere
\PassOptionsToPackage{unicode}{hyperref}
\PassOptionsToPackage{hyphens}{url}
\PassOptionsToPackage{dvipsnames,svgnames,x11names}{xcolor}
%
\documentclass[
  letterpaper,
  DIV=11,
  numbers=noendperiod]{scrartcl}
\usepackage{xcolor}
\usepackage{amsmath,amssymb}
\setcounter{secnumdepth}{-\maxdimen} % remove section numbering
\usepackage{iftex}
\ifPDFTeX
  \usepackage[T1]{fontenc}
  \usepackage[utf8]{inputenc}
  \usepackage{textcomp} % provide euro and other symbols
\else % if luatex or xetex
  \usepackage{unicode-math} % this also loads fontspec
  \defaultfontfeatures{Scale=MatchLowercase}
  \defaultfontfeatures[\rmfamily]{Ligatures=TeX,Scale=1}
\fi
\usepackage{lmodern}
\ifPDFTeX\else
  % xetex/luatex font selection
\fi
% Use upquote if available, for straight quotes in verbatim environments
\IfFileExists{upquote.sty}{\usepackage{upquote}}{}
\IfFileExists{microtype.sty}{% use microtype if available
  \usepackage[]{microtype}
  \UseMicrotypeSet[protrusion]{basicmath} % disable protrusion for tt fonts
}{}
\makeatletter
\@ifundefined{KOMAClassName}{% if non-KOMA class
  \IfFileExists{parskip.sty}{%
    \usepackage{parskip}
  }{% else
    \setlength{\parindent}{0pt}
    \setlength{\parskip}{6pt plus 2pt minus 1pt}}
}{% if KOMA class
  \KOMAoptions{parskip=half}}
\makeatother
% Make \paragraph and \subparagraph free-standing
\makeatletter
\ifx\paragraph\undefined\else
  \let\oldparagraph\paragraph
  \renewcommand{\paragraph}{
    \@ifstar
      \xxxParagraphStar
      \xxxParagraphNoStar
  }
  \newcommand{\xxxParagraphStar}[1]{\oldparagraph*{#1}\mbox{}}
  \newcommand{\xxxParagraphNoStar}[1]{\oldparagraph{#1}\mbox{}}
\fi
\ifx\subparagraph\undefined\else
  \let\oldsubparagraph\subparagraph
  \renewcommand{\subparagraph}{
    \@ifstar
      \xxxSubParagraphStar
      \xxxSubParagraphNoStar
  }
  \newcommand{\xxxSubParagraphStar}[1]{\oldsubparagraph*{#1}\mbox{}}
  \newcommand{\xxxSubParagraphNoStar}[1]{\oldsubparagraph{#1}\mbox{}}
\fi
\makeatother


\usepackage{longtable,booktabs,array}
\usepackage{calc} % for calculating minipage widths
% Correct order of tables after \paragraph or \subparagraph
\usepackage{etoolbox}
\makeatletter
\patchcmd\longtable{\par}{\if@noskipsec\mbox{}\fi\par}{}{}
\makeatother
% Allow footnotes in longtable head/foot
\IfFileExists{footnotehyper.sty}{\usepackage{footnotehyper}}{\usepackage{footnote}}
\makesavenoteenv{longtable}
\usepackage{graphicx}
\makeatletter
\newsavebox\pandoc@box
\newcommand*\pandocbounded[1]{% scales image to fit in text height/width
  \sbox\pandoc@box{#1}%
  \Gscale@div\@tempa{\textheight}{\dimexpr\ht\pandoc@box+\dp\pandoc@box\relax}%
  \Gscale@div\@tempb{\linewidth}{\wd\pandoc@box}%
  \ifdim\@tempb\p@<\@tempa\p@\let\@tempa\@tempb\fi% select the smaller of both
  \ifdim\@tempa\p@<\p@\scalebox{\@tempa}{\usebox\pandoc@box}%
  \else\usebox{\pandoc@box}%
  \fi%
}
% Set default figure placement to htbp
\def\fps@figure{htbp}
\makeatother





\setlength{\emergencystretch}{3em} % prevent overfull lines

\providecommand{\tightlist}{%
  \setlength{\itemsep}{0pt}\setlength{\parskip}{0pt}}



 


% TODO: Add custom LaTeX header directives here
\KOMAoption{captions}{tableheading}
\makeatletter
\@ifpackageloaded{caption}{}{\usepackage{caption}}
\AtBeginDocument{%
\ifdefined\contentsname
  \renewcommand*\contentsname{Table of contents}
\else
  \newcommand\contentsname{Table of contents}
\fi
\ifdefined\listfigurename
  \renewcommand*\listfigurename{List of Figures}
\else
  \newcommand\listfigurename{List of Figures}
\fi
\ifdefined\listtablename
  \renewcommand*\listtablename{List of Tables}
\else
  \newcommand\listtablename{List of Tables}
\fi
\ifdefined\figurename
  \renewcommand*\figurename{Figure}
\else
  \newcommand\figurename{Figure}
\fi
\ifdefined\tablename
  \renewcommand*\tablename{Table}
\else
  \newcommand\tablename{Table}
\fi
}
\@ifpackageloaded{float}{}{\usepackage{float}}
\floatstyle{ruled}
\@ifundefined{c@chapter}{\newfloat{codelisting}{h}{lop}}{\newfloat{codelisting}{h}{lop}[chapter]}
\floatname{codelisting}{Listing}
\newcommand*\listoflistings{\listof{codelisting}{List of Listings}}
\makeatother
\makeatletter
\makeatother
\makeatletter
\@ifpackageloaded{caption}{}{\usepackage{caption}}
\@ifpackageloaded{subcaption}{}{\usepackage{subcaption}}
\makeatother
\usepackage{bookmark}
\IfFileExists{xurl.sty}{\usepackage{xurl}}{} % add URL line breaks if available
\urlstyle{same}
\hypersetup{
  pdftitle={GeoLSTM},
  pdfauthor={Gerard Franco-Panadés; Eliza Dealloc},
  pdfkeywords={Spatio-temporal traffic prediction, Road geometry
integration, Heavy-weight vehicle intensity, Neural networks, Long
Short-Term Memory (LSTM), Mid-term traffic prediction},
  colorlinks=true,
  linkcolor={blue},
  filecolor={Maroon},
  citecolor={Blue},
  urlcolor={Blue},
  pdfcreator={LaTeX via pandoc}}


\title{GeoLSTM}
\author{Gerard Franco-Panadés \and Eliza Dealloc}
\date{}
\begin{document}
\maketitle
\begin{abstract}
Accurate prediction of dynamic states in complex physical systems is
critical for intelligent decision-support and control. This paper
presents a novel spatio-temporal deep learning framework tailored to
expert systems operating in mobility and transportation environments.
Using a large-scale, multi-source dataset collected over two years from
a 100-kilometer highway segment, the proposed model integrates static
road geometry features---specifically curvature and slope---into a Long
Short-Term Memory (LSTM) architecture. We introduce GeoLSTM, a
geometry-aware neural network model that enhances mid-term traffic
forecasting accuracy across multiple metrics, including vehicle speed,
total intensity and heavy-weight vehicle intensity, addressing the
common omission of static road geometry by embedding curvature and
slope. Our experiments demonstrate that incorporating static spatial
features significantly reduces prediction error (up to 26\% for speed
and 16\% for heavy-vehicle intensity) and improves model stability,
outperforming both traditional baselines and transformer-based
alternatives. We further quantify computational costs to assess
real-time deployment feasibility. These results establish a robust
methodology for integrating contextual spatial data into deep
learning-based expert systems, with potential generalization across
other domains requiring predictive modelling in structured, data-rich
environments.
\end{abstract}

\renewcommand*\contentsname{Table of contents}
{
\hypersetup{linkcolor=}
\setcounter{tocdepth}{3}
\tableofcontents
}

\subsection{Introduction}\label{sec-intro}

\emph{TODO} In recent years, traffic management systems are increasingly
depending on precise real-time traffic forecasting (Pan et al., 2024) to
improve response effectiveness and resource allocation when exceptional
situations like accidents or congestion happen. For these systems to
effectively handle regular traffic patterns, predictive accuracy is
crucial. Furthermore, the surge of new technologies related to smart
cities infrastructure, autonomous vehicles (AVs) or mobility services
also benefit from new data sources for accurate traffic state
predictions. Nowadays dynamic inputs from high-resolution traffic
sensors and GPS data allow for more responsive and adaptable road
network control in a variety of situations (Seo et al., 2017). However,
most operational forecasting systems still neglect the static physical
constraints of the road network, such as curvature and slope, which can
systematically influence vehicle behaviour, lane capacity, and
congestion onset. This motivates our integration of road geometry into
deep learning architectures, aiming to improve both accuracy and
stability in real-time traffic prediction. The data used in this study
comes from the AP-7 highway in Catalonia, a high-traffic corridor that
serves as a representative example for evaluating the proposed approach.
The methodology, combining static geometry with dynamic traffic
features, is designed to be applicable to other highway networks with
similar data availability, enabling broader deployment and
generalization. Covering a 100-kilometer segment, the highway has
infrastructure sensors (inductive loop detectors) providing detailed
traffic intensity data (including heavy vehicles) and GPS-derived speed
statistics at a granularity of 1-minute intervals. This extensive
two-year dataset was aggregated to hourly intervals, facilitating
precise and operationally meaningful mid-term traffic forecasts. This
particular segment was selected due to its high frequency of
congestion-related incidents, which significantly impact regional
mobility and safety (Guasch, 2024). The scope of this article is
interurban roads (highways), with urban environments outside the present
scope because they must be treated differently (Zhu et al., 2023). The
core contribution of this research lies in the applied integration of
static road geometry features---specifically curvature and slope---into
established deep learning architectures for traffic forecasting. Rather
than introducing new architecture types, this work focuses on model
improvements for real-world deployment in structured highway
environments, substantially improving predictive accuracy. ``Existing
deep learning approaches often lack spatial awareness beyond immediate
sensor locations, and even spatial models rarely incorporate static
environmental features .This results in forecasts that may perform well
under stable conditions but degrade when geometric constraints alter
driver behaviour. Our work directly addresses this shortcoming (Wang et
al., 2024). Incorporating additional contextual variables, such as
vehicle weight classifications and a mobility calendar indicating
holidays and special events, further enhances the system's performance.
Given the primary objectives of the Catalan Traffic
Authority---improving road safety and reducing mortality rates---the
accuracy gains offered by this enhanced forecasting approach directly
translate into practical operational improvements. By developing
real-time traffic forecasting systems, traffic authorities can
proactively implement targeted interventions to mitigate potential
hazards or congestion, thus significantly enhancing regional traffic
safety and operational efficiency. This practical application
underscores the broader relevance of this research in developing
intelligent, data-driven decision-support systems capable of improving
transportation efficiency, safety, and reliability. By combining
detailed road geometry data with dynamic traffic compositions, the
proposed model accurately captures complex traffic patterns across
diverse road conditions, setting a new benchmark for predictive accuracy
and operational utility in intelligent transportation systems (Seo et
al., 2017).

The main contributions of this work are as follows: • We propose
GeoLSTM, a geometry-aware LSTM model that integrates static road
features (curvature, slope) alongside dynamic traffic variables for
spatio-temporal mid-term highway traffic forecasting. • Demonstration
that the proposed GeoLSTM improves traffic predictions accuracy by up to
26\% compared to geometry-agnostic models. • Performance comparison
showing that GeoLSTM outperforms advanced baselines, including
Transformer-based architectures for highway traffic hourly data
providing mid-term predictions. • Validation of the model on a
large-scale, real-world dataset collected over two years on a 100-km
highway, demonstrating significant error reductions across multiple
traffic indicators. • Evidence of enhanced spatial robustness across
diverse highway segments. • Reproducible methodology using open-access
road geometry data, enabling adaptation to other highway networks with
similar data availability. The remainder of this article is structured
as follows. Section 2 provides a comprehensive review of prior work in
traffic forecasting, spanning from traditional statistical models to
advanced deep learning architectures. Section 3 details the dataset and
formalizes the forecasting problem, including the spatial and temporal
structure of the traffic data and the inclusion of static geometric
features. Section 4 describes the methodological framework, outlining
the neural architectures considered, the integration of road geometry,
and the training and optimization protocols. Section 5 presents the
experimental results, including comparative performance analyses,
feature importance assessments, and computational efficiency
evaluations. Finally, Section 6 discusses the broader implications of
the findings, acknowledges limitations, and proposes directions for
future research.




\end{document}
